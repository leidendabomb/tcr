\documentclass[10pt]{article}

\frenchspacing
\usepackage[a4paper,margin=2cm]{geometry}
\usepackage[english]{babel}
\usepackage{listings}
\usepackage{fancyhdr}
\usepackage{amssymb, amsmath} 
\usepackage{hyperref}
\pagestyle{fancy}
\renewcommand{\headrulewidth}{0.1pt}
\lhead{Universiteit Leiden}
\rhead{\thepage}
\fancyfoot{}
\lstset{
  breakatwhitespace=true,
  breaklines=true,
  tabsize=2,
  basicstyle=\small,
  numbers=left,
  numberstyle=\tiny,
  numberfirstline=false,
  stepnumber=1,
  identifierstyle=\ttfamily
}

\title{Mad hacks\\[1cm]
\small{Team Reference Document of:}}
\author{:()\{:$\vert$:\&\};: (da bomb)\\[1cm]
	\small{Universiteit Leiden}}

\begin{document}

\fontsize{10}{12}

\selectlanguage{english}

\maketitle

\newpage

\tableofcontents

\section{Team}
\begin{itemize}
\item Tobias de Jong
\item Bert Peters
\item Bart van Strien
\end{itemize}

\section{Confs}

\subsection{vimrc}
\lstinputlisting{vimrc}

\subsection{makefile}
\lstinputlisting[language=make]{Makefile}

\lstset{language=c++}

\subsection{base.cpp}
\lstinputlisting[language=c++]{base.cpp}

\section{Algorithms}
\subsection{List}
\begin{itemize}
\item Depth First Search
\item Breadth First Search
\item Binary Search
\item Linear search
\item Dijkstra
\begin{itemize}
\item Bellman-Ford- Dijkstra for negative weights. relax $v-1$ times for all edges
\end{itemize}
\item Ford-Fulkerson: max flow - find paths from source to sink while possible. (breadth- or depth first)
\begin{itemize}
\item Max flow - min cut: FF- then extend from source until no edges with non-zero residual capacity can be added.
\item bipartite matching - find max matches out of a possible set by converting to max flow
\end{itemize}
\item Quicksort $O(n^2)$ - pick pivot, $A<pivot \leq B$, effort splitting
\item Mergesort $O(n \log n)$ - effort merging, compare first of A with first of B
\item Heapsort $O(n \log n)$ - turn array into heap, then repeatedly get the largest value
\item Bubble sort $O(n^2)$- swap neighbours
\item Insertion sort $O(n^2)$
\item Select sort $O(n^2)$ - select minimum in unsorted rest
\item Bucket sort $O(n^2)$ - buckets contain elements with values $a < x \leq b$
\item Radix sort $O(nk)$ - sort numbers by least significant digit (radix $k$), to highest significant digit with stable (typically counting) sort
\end{itemize}

\subsection{Dijkstra}
\lstinputlisting[language=c++]{dijkstra.cpp}

\subsection{Floodfill}
\lstinputlisting[language=c++]{floodfill.cpp}

\subsection{Building heap}
Insert at the bottom, bubble up by swapping with parent if bigger.

\subsection{Algorithms}
\begin{itemize}
\item Brute Force
\item Backtracking
\item Divide and Conquer - split and recurse \\
	Depth of Tree
\item Dynamic Programming - fill array bottom up \\
	Fibonacci, knapsack
\item Memoization - fill array as you go
\item Greedy shit
\item Dijkstra
\item Branch-and-bound - optimization \\
	Cut off when estimate below minimum or above maximum, travelling salesman
\item Transform-and-conquer - transform to known problem, solve that, transform answer
\end{itemize}

\section{Math}
\subsection{(co)sinusregel}

\[\frac{\sin a}{\alpha} = \frac{\sin b}{\beta} =\frac{\sin c}{\gamma}\]
\[c^2=a^2+b^2-2ab\cos(\gamma)\]

\subsection{Combinatorics}

\[\binom{n+m}{r} = \sum_{k=0}^r \binom{n}{k} \binom{m}{r-k}\]

\subsection{Deelbaarheid}

\subsubsection{3 en 9}

De som van de cijfers moet deelbaar zijn door respectievelijk 3 en 9.

\begin{align*}
	291261 \to & 19 \text{ \emph{niet deelbaar}} \\
	815308974 \to & 45 \text{ \emph{wel deelbaar}}
\end{align*}

\subsubsection{7}

Drie herhaalbare stappen, ge\"illustreerd aan de hand van het voorbeeld 203.

\begin{enumerate}
	\item Neem het laatste cijfer van het oorspronkelijke getal en verdubbel het, in het voorbeeld $3 \to 6$.
	\item Haal dit van het resterende geval, in het voorbeeld $20 - 6 = 14$.
	\item Is het duidelijk dat dit deelbaar is door zeven? Zo niet, ga naar stap 1.
\end{enumerate}

\subsubsection{11}

De alternerende som van de cijfers moet gelijk zijn aan 0. Dit is duidelijker met een voorbeeld.

\[123455216 \to 1 - 2 + 3 - 4 + 5 - 5 + 2 - 1 + 0 = 0\]

\end{document}
